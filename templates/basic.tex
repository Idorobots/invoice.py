\documentclass[11pt]{article}
\usepackage[margin=2cm]{geometry}
\usepackage[T1]{fontenc}
\usepackage{fontspec}
\usepackage{xunicode}
\usepackage{xltxtra}
\usepackage{fixltx2e}
\usepackage{longtable}
\usepackage[parfill]{parskip}
\pagestyle{empty}

\begin{document}

\begin{center}
  \Large
  \textbf{Invoice \#{{invoice_number}}}
\end{center}

\textbf{Issuer:} \\
       {{issuer.name}} \\
       {{issuer.address}} \\
       {{issuer.numbers}} \\
       bank account: {{issuer.bank}} \\
       {{issuer.extra}}

\textbf{Recipient:} \\
       {{recipient.name}} \\
       {{recipient.address}} \\
       {{recipient.numbers}} \\
       {{recipient.extra}}

\textbf{Date of sale:} {{date_of_sale}} \\
\textbf{Date of issue:} {{issue_date}}

\newcounter{ItemIndex}
\newcommand{\itemindex}{\stepcounter{ItemIndex} \theItemIndex}

\newcounter{TaxIndex}

\begin{center}
  \small
  \begin{tabular}{rlrrlrrrr}
    \# & title & net price & qt. & unit & net amount & VAT [\%] & VAT [{{currency}}] & amount \\
    \hline
    {{#items}}
     \itemindex & {{name}} & {{net_price}} & {{quantity}} & {{quantity_unit}} & {{net_amount}} & {{vat}} & {{vat_amount}} & {{amount}} \\
    {{/items}}

    \hline
    & \textbf{total:} &  &  &  & {{total_net_amount}} & - & {{total_vat_amount}} & {{total_amount}} \\
    \hline
    {{#taxes}}
      & \ifnum \theTaxIndex=0 \textbf{including:}\stepcounter{TaxIndex} \fi &  &  &  & {{total_net_amount}} & {{vat}} & {{total_vat_amount}} & {{total_amount}} \\
    {{/taxes}}
  \end{tabular}
\end{center}

\textbf{Total:} {{total_amount}} {{currency}} \\
\textbf{Due date:} {{due_date}} \\
\textbf{Payment to:} {{issuer.bank}}

\begin{center}
  \tiny
  {{uuid}}
\end{center}
\end{document}
